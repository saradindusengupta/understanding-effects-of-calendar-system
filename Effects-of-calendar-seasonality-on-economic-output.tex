% Options for packages loaded elsewhere
\PassOptionsToPackage{unicode}{hyperref}
\PassOptionsToPackage{hyphens}{url}
%
\documentclass[
]{article}
\usepackage{amsmath,amssymb}
\usepackage{iftex}
\ifPDFTeX
  \usepackage[T1]{fontenc}
  \usepackage[utf8]{inputenc}
  \usepackage{textcomp} % provide euro and other symbols
\else % if luatex or xetex
  \usepackage{unicode-math} % this also loads fontspec
  \defaultfontfeatures{Scale=MatchLowercase}
  \defaultfontfeatures[\rmfamily]{Ligatures=TeX,Scale=1}
\fi
\usepackage{lmodern}
\ifPDFTeX\else
  % xetex/luatex font selection
\fi
% Use upquote if available, for straight quotes in verbatim environments
\IfFileExists{upquote.sty}{\usepackage{upquote}}{}
\IfFileExists{microtype.sty}{% use microtype if available
  \usepackage[]{microtype}
  \UseMicrotypeSet[protrusion]{basicmath} % disable protrusion for tt fonts
}{}
\makeatletter
\@ifundefined{KOMAClassName}{% if non-KOMA class
  \IfFileExists{parskip.sty}{%
    \usepackage{parskip}
  }{% else
    \setlength{\parindent}{0pt}
    \setlength{\parskip}{6pt plus 2pt minus 1pt}}
}{% if KOMA class
  \KOMAoptions{parskip=half}}
\makeatother
\usepackage{xcolor}
\usepackage[margin=1in]{geometry}
\usepackage{graphicx}
\makeatletter
\def\maxwidth{\ifdim\Gin@nat@width>\linewidth\linewidth\else\Gin@nat@width\fi}
\def\maxheight{\ifdim\Gin@nat@height>\textheight\textheight\else\Gin@nat@height\fi}
\makeatother
% Scale images if necessary, so that they will not overflow the page
% margins by default, and it is still possible to overwrite the defaults
% using explicit options in \includegraphics[width, height, ...]{}
\setkeys{Gin}{width=\maxwidth,height=\maxheight,keepaspectratio}
% Set default figure placement to htbp
\makeatletter
\def\fps@figure{htbp}
\makeatother
\setlength{\emergencystretch}{3em} % prevent overfull lines
\providecommand{\tightlist}{%
  \setlength{\itemsep}{0pt}\setlength{\parskip}{0pt}}
\setcounter{secnumdepth}{-\maxdimen} % remove section numbering
\newlength{\cslhangindent}
\setlength{\cslhangindent}{1.5em}
\newlength{\csllabelwidth}
\setlength{\csllabelwidth}{3em}
\newlength{\cslentryspacingunit} % times entry-spacing
\setlength{\cslentryspacingunit}{\parskip}
\newenvironment{CSLReferences}[2] % #1 hanging-ident, #2 entry spacing
 {% don't indent paragraphs
  \setlength{\parindent}{0pt}
  % turn on hanging indent if param 1 is 1
  \ifodd #1
  \let\oldpar\par
  \def\par{\hangindent=\cslhangindent\oldpar}
  \fi
  % set entry spacing
  \setlength{\parskip}{#2\cslentryspacingunit}
 }%
 {}
\usepackage{calc}
\newcommand{\CSLBlock}[1]{#1\hfill\break}
\newcommand{\CSLLeftMargin}[1]{\parbox[t]{\csllabelwidth}{#1}}
\newcommand{\CSLRightInline}[1]{\parbox[t]{\linewidth - \csllabelwidth}{#1}\break}
\newcommand{\CSLIndent}[1]{\hspace{\cslhangindent}#1}
\ifLuaTeX
  \usepackage{selnolig}  % disable illegal ligatures
\fi
\IfFileExists{bookmark.sty}{\usepackage{bookmark}}{\usepackage{hyperref}}
\IfFileExists{xurl.sty}{\usepackage{xurl}}{} % add URL line breaks if available
\urlstyle{same}
\hypersetup{
  pdftitle={Effects of calendar seasonality in Indian calendar system},
  hidelinks,
  pdfcreator={LaTeX via pandoc}}

\title{Effects of calendar seasonality in Indian calendar system}
\author{}
\date{\vspace{-2.5em}}

\begin{document}
\maketitle

\hypertarget{effects-of-calendar-seasonality-on-economic-output}{%
\section{Effects of calendar seasonality on economic
output}\label{effects-of-calendar-seasonality-on-economic-output}}

Calendar have enormous effect on economic, social and cultural behaviors
but nowhere it is profound than a country or region's economic output
indicators such as industrial output, CPI, stock-market transaction,
import-exports but in a globalized world, all of the economic
time-series modelling is modeled around Gregorian calendar system. In a
country like India with multiple regions and festivals from multiple
religion as well, a single calendar will not be sufficient to model
various seasonal component. For example, during the festival month of
Diwali, consumer consumption rate is usually higher and during the
month, industrial output is generally lower.

\hypertarget{previous-work}{%
\subsubsection{Previous Work}\label{previous-work}}

There have been significant work published already to identify and
remove multiple lagged seasonality from time-series (De Livera, Hyndman,
and Snyder 2011) considering technical papers published from The US
Bureau of Census\footnote{\url{https://www.google.co.in/books/edition/The_X_11_Variant_of_the_Census_Method_II/BFIfiGmatUoC?hl=en\&gbpv=0\&kptab=overview}}.
There also have been significant work done to identify and remove
seasonality, especially concerning religious festival based on Gregorian
calendar by the Bank of Spain(Maravall and Sánchez 2000). Similar work
is present for lunar\footnote{\url{https://www.census.gov/library/working-papers/2002/adrm/lin-01.html}}
and luni-solar\footnote{\url{https://assets.cambridge.org/97811070/57623/frontmatter/9781107057623_frontmatter.pdf}}
based calendar system as well. Similar work in the context of Indian
seasonality effect is also present\footnote{\url{https://www.nipfp.org.in/media/medialibrary/2016/01/WP_2016_160.pdf}}\footnote{\url{https://www.census.gov/content/dam/Census/library/working-papers/2017/adrm/rrs2017-04.pdf}}\footnote{\url{https://journals.indexcopernicus.com/api/file/viewByFileId/690045.pdf}}.

\hypertarget{a-comparative-analysis-on-calendar-seasonality-with-indian-calendar-system}{%
\subsubsection{A comparative Analysis on calendar seasonality with
Indian calendar
system}\label{a-comparative-analysis-on-calendar-seasonality-with-indian-calendar-system}}

A comparative study was performed to understand calendar seasonality
creeping in while modelling seasonal decomposition on time-series
objects. especially on econometric data-sets. To understand if any
methodologies can be obtained and to provide a better, quantitative
understanding of the seasonality, India's Industrial output after the
year 2000 and modelling holiday seasonality with Diwali dates
corresponding to that. To understand calendar seasonality, dates of the
festival of Diwali were obtained from Sax and Eddelbuettel (2018)
package and the dates were converted to Vedic calendar dates. Two
different time-series were modeled, one with Gregorian calendar dates
and another with Vedic calendar dates. A corresponding time-series
object of India's industrial output from Sax and Eddelbuettel (2018)
package was also converted to Vedic calendar based time-series and
scale. In the both the cases, a seasonal decomposition was observed
visible along the red line and \texttt{X-13\ ARIMA\ SEATS} was used for
seasonality decomposition. Due to change in type of calendar, missing
values were omitted to keep the frequency of the time-series same for
both calendar systems. Bokde et al. (2022) was used for the all the date
conversio and conversion of time-series objects in the comparative
study.

\hypertarget{future-work}{%
\subsubsection{Future Work}\label{future-work}}

Since, there is no comprehensive work done on understanding various
seasonal component of the Indian calendar system and its effect on
economic output metrics, it would be pertinent to pursue work on
identifying various monthly and quarterly seasonal component.

Moreover, the existing developed package \texttt{VedicDateTime}, needs
to be updated with dates of significant festivals of the Vedic calendars
which drives economic activities.

\hypertarget{refs}{}
\begin{CSLReferences}{1}{0}
\leavevmode\vadjust pre{\hypertarget{ref-VedicDateTime}{}}%
Bokde, Neeraj Dhanraj, Prajwal Kailasnath Patil, Saradindu Sengupta, and
Andrés Elías Feijóo Lorenzo. 2022. {``VedicDateTime: Vedic Calendar
System.''} \url{https://CRAN.R-project.org/package=VedicDateTime}.

\leavevmode\vadjust pre{\hypertarget{ref-delivera2011}{}}%
De Livera, Alysha M., Rob J. Hyndman, and Ralph D. Snyder. 2011.
{``Forecasting Time Series With Complex Seasonal Patterns Using
Exponential Smoothing.''} \emph{Journal of the American Statistical
Association} 106 (496): 1513--27.
\url{https://doi.org/10.1198/jasa.2011.tm09771}.

\leavevmode\vadjust pre{\hypertarget{ref-maravall2000}{}}%
Maravall, Agustín, and Fernando J. Sánchez. 2000. {``An Application of
TRAMO-SEATS; Model Selection and Out-of-Sample Performance. The Swiss
CPI Series.''} In, 121--30. Physica-Verlag HD.
\url{https://doi.org/10.1007/978-3-642-57678-2_11}.

\leavevmode\vadjust pre{\hypertarget{ref-seasonal}{}}%
Sax, Christoph, and Dirk Eddelbuettel. 2018. {``Seasonal Adjustment by
{\textbraceleft}x-13ARIMA-SEATS{\textbraceright} in
{\textbraceleft}r{\textbraceright}''} 87.
\url{https://doi.org/10.18637/jss.v087.i11}.

\end{CSLReferences}

\end{document}
