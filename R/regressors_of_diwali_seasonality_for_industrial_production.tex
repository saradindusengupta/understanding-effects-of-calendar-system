% Options for packages loaded elsewhere
\PassOptionsToPackage{unicode}{hyperref}
\PassOptionsToPackage{hyphens}{url}
%
\documentclass[
]{article}
\usepackage{amsmath,amssymb}
\usepackage{iftex}
\ifPDFTeX
  \usepackage[T1]{fontenc}
  \usepackage[utf8]{inputenc}
  \usepackage{textcomp} % provide euro and other symbols
\else % if luatex or xetex
  \usepackage{unicode-math} % this also loads fontspec
  \defaultfontfeatures{Scale=MatchLowercase}
  \defaultfontfeatures[\rmfamily]{Ligatures=TeX,Scale=1}
\fi
\usepackage{lmodern}
\ifPDFTeX\else
  % xetex/luatex font selection
\fi
% Use upquote if available, for straight quotes in verbatim environments
\IfFileExists{upquote.sty}{\usepackage{upquote}}{}
\IfFileExists{microtype.sty}{% use microtype if available
  \usepackage[]{microtype}
  \UseMicrotypeSet[protrusion]{basicmath} % disable protrusion for tt fonts
}{}
\makeatletter
\@ifundefined{KOMAClassName}{% if non-KOMA class
  \IfFileExists{parskip.sty}{%
    \usepackage{parskip}
  }{% else
    \setlength{\parindent}{0pt}
    \setlength{\parskip}{6pt plus 2pt minus 1pt}}
}{% if KOMA class
  \KOMAoptions{parskip=half}}
\makeatother
\usepackage{xcolor}
\usepackage[margin=1in]{geometry}
\usepackage{color}
\usepackage{fancyvrb}
\newcommand{\VerbBar}{|}
\newcommand{\VERB}{\Verb[commandchars=\\\{\}]}
\DefineVerbatimEnvironment{Highlighting}{Verbatim}{commandchars=\\\{\}}
% Add ',fontsize=\small' for more characters per line
\usepackage{framed}
\definecolor{shadecolor}{RGB}{248,248,248}
\newenvironment{Shaded}{\begin{snugshade}}{\end{snugshade}}
\newcommand{\AlertTok}[1]{\textcolor[rgb]{0.94,0.16,0.16}{#1}}
\newcommand{\AnnotationTok}[1]{\textcolor[rgb]{0.56,0.35,0.01}{\textbf{\textit{#1}}}}
\newcommand{\AttributeTok}[1]{\textcolor[rgb]{0.13,0.29,0.53}{#1}}
\newcommand{\BaseNTok}[1]{\textcolor[rgb]{0.00,0.00,0.81}{#1}}
\newcommand{\BuiltInTok}[1]{#1}
\newcommand{\CharTok}[1]{\textcolor[rgb]{0.31,0.60,0.02}{#1}}
\newcommand{\CommentTok}[1]{\textcolor[rgb]{0.56,0.35,0.01}{\textit{#1}}}
\newcommand{\CommentVarTok}[1]{\textcolor[rgb]{0.56,0.35,0.01}{\textbf{\textit{#1}}}}
\newcommand{\ConstantTok}[1]{\textcolor[rgb]{0.56,0.35,0.01}{#1}}
\newcommand{\ControlFlowTok}[1]{\textcolor[rgb]{0.13,0.29,0.53}{\textbf{#1}}}
\newcommand{\DataTypeTok}[1]{\textcolor[rgb]{0.13,0.29,0.53}{#1}}
\newcommand{\DecValTok}[1]{\textcolor[rgb]{0.00,0.00,0.81}{#1}}
\newcommand{\DocumentationTok}[1]{\textcolor[rgb]{0.56,0.35,0.01}{\textbf{\textit{#1}}}}
\newcommand{\ErrorTok}[1]{\textcolor[rgb]{0.64,0.00,0.00}{\textbf{#1}}}
\newcommand{\ExtensionTok}[1]{#1}
\newcommand{\FloatTok}[1]{\textcolor[rgb]{0.00,0.00,0.81}{#1}}
\newcommand{\FunctionTok}[1]{\textcolor[rgb]{0.13,0.29,0.53}{\textbf{#1}}}
\newcommand{\ImportTok}[1]{#1}
\newcommand{\InformationTok}[1]{\textcolor[rgb]{0.56,0.35,0.01}{\textbf{\textit{#1}}}}
\newcommand{\KeywordTok}[1]{\textcolor[rgb]{0.13,0.29,0.53}{\textbf{#1}}}
\newcommand{\NormalTok}[1]{#1}
\newcommand{\OperatorTok}[1]{\textcolor[rgb]{0.81,0.36,0.00}{\textbf{#1}}}
\newcommand{\OtherTok}[1]{\textcolor[rgb]{0.56,0.35,0.01}{#1}}
\newcommand{\PreprocessorTok}[1]{\textcolor[rgb]{0.56,0.35,0.01}{\textit{#1}}}
\newcommand{\RegionMarkerTok}[1]{#1}
\newcommand{\SpecialCharTok}[1]{\textcolor[rgb]{0.81,0.36,0.00}{\textbf{#1}}}
\newcommand{\SpecialStringTok}[1]{\textcolor[rgb]{0.31,0.60,0.02}{#1}}
\newcommand{\StringTok}[1]{\textcolor[rgb]{0.31,0.60,0.02}{#1}}
\newcommand{\VariableTok}[1]{\textcolor[rgb]{0.00,0.00,0.00}{#1}}
\newcommand{\VerbatimStringTok}[1]{\textcolor[rgb]{0.31,0.60,0.02}{#1}}
\newcommand{\WarningTok}[1]{\textcolor[rgb]{0.56,0.35,0.01}{\textbf{\textit{#1}}}}
\usepackage{graphicx}
\makeatletter
\def\maxwidth{\ifdim\Gin@nat@width>\linewidth\linewidth\else\Gin@nat@width\fi}
\def\maxheight{\ifdim\Gin@nat@height>\textheight\textheight\else\Gin@nat@height\fi}
\makeatother
% Scale images if necessary, so that they will not overflow the page
% margins by default, and it is still possible to overwrite the defaults
% using explicit options in \includegraphics[width, height, ...]{}
\setkeys{Gin}{width=\maxwidth,height=\maxheight,keepaspectratio}
% Set default figure placement to htbp
\makeatletter
\def\fps@figure{htbp}
\makeatother
\setlength{\emergencystretch}{3em} % prevent overfull lines
\providecommand{\tightlist}{%
  \setlength{\itemsep}{0pt}\setlength{\parskip}{0pt}}
\setcounter{secnumdepth}{-\maxdimen} % remove section numbering
\newlength{\cslhangindent}
\setlength{\cslhangindent}{1.5em}
\newlength{\csllabelwidth}
\setlength{\csllabelwidth}{3em}
\newlength{\cslentryspacingunit} % times entry-spacing
\setlength{\cslentryspacingunit}{\parskip}
\newenvironment{CSLReferences}[2] % #1 hanging-ident, #2 entry spacing
 {% don't indent paragraphs
  \setlength{\parindent}{0pt}
  % turn on hanging indent if param 1 is 1
  \ifodd #1
  \let\oldpar\par
  \def\par{\hangindent=\cslhangindent\oldpar}
  \fi
  % set entry spacing
  \setlength{\parskip}{#2\cslentryspacingunit}
 }%
 {}
\usepackage{calc}
\newcommand{\CSLBlock}[1]{#1\hfill\break}
\newcommand{\CSLLeftMargin}[1]{\parbox[t]{\csllabelwidth}{#1}}
\newcommand{\CSLRightInline}[1]{\parbox[t]{\linewidth - \csllabelwidth}{#1}\break}
\newcommand{\CSLIndent}[1]{\hspace{\cslhangindent}#1}
\ifLuaTeX
  \usepackage{selnolig}  % disable illegal ligatures
\fi
\IfFileExists{bookmark.sty}{\usepackage{bookmark}}{\usepackage{hyperref}}
\IfFileExists{xurl.sty}{\usepackage{xurl}}{} % add URL line breaks if available
\urlstyle{same}
\hypersetup{
  pdftitle={Regressors of Diwali on Industrial Production of India},
  hidelinks,
  pdfcreator={LaTeX via pandoc}}

\title{Regressors of Diwali on Industrial Production of India}
\author{}
\date{\vspace{-2.5em}}

\begin{document}
\maketitle

Diwali is the most important festival of India and the timing of its
distorts the monthly time-series of industrial production heavily.
Generally Diwali is celebrated in the month of October according to
Gregorian Calendar but that is not fixed and depending on which month it
is celebrated the industrial production index also changes. The standard
software packages for seasonal adjustment, \texttt{X-12-ARIMA} and
\texttt{X-13-ARIMA-SEATS} (developed by the U.S. Census Bureau) or Tramo
Seats (developed by the Bank of Spain) have a built-in adjustment
procedure for Easter holiday, but not for Diwali. However, all packages
allow for the inclusion of user defined variables, and the Chinese New
Year can be modeled as such. \texttt{seasonal} (Sax and Eddelbuettel
2018) is an interface to X-13ARIMA-SEATS.

\begin{Shaded}
\begin{Highlighting}[]
\FunctionTok{rm}\NormalTok{(}\AttributeTok{list =} \FunctionTok{ls}\NormalTok{())}
\FunctionTok{library}\NormalTok{(seasonal)}
\FunctionTok{library}\NormalTok{(VedicDateTime)}
\FunctionTok{library}\NormalTok{(insol)}
\FunctionTok{library}\NormalTok{(forecast)}
\end{Highlighting}
\end{Shaded}

\begin{verbatim}
## Registered S3 method overwritten by 'quantmod':
##   method            from
##   as.zoo.data.frame zoo
\end{verbatim}

\begin{Shaded}
\begin{Highlighting}[]
\FunctionTok{library}\NormalTok{(zoo)}
\end{Highlighting}
\end{Shaded}

\begin{verbatim}
## 
## Attaching package: 'zoo'
\end{verbatim}

\begin{verbatim}
## The following objects are masked from 'package:base':
## 
##     as.Date, as.Date.numeric
\end{verbatim}

\begin{Shaded}
\begin{Highlighting}[]
\NormalTok{knitr}\SpecialCharTok{::}\NormalTok{opts\_chunk}\SpecialCharTok{$}\FunctionTok{set}\NormalTok{(}\AttributeTok{warning =} \ConstantTok{FALSE}\NormalTok{, }\AttributeTok{message =} \ConstantTok{FALSE}\NormalTok{)}
\FunctionTok{data}\NormalTok{(seasonal)}
\FunctionTok{data}\NormalTok{(holiday)}
\end{Highlighting}
\end{Shaded}

\hypertarget{considering-industrial-production-of-india-after-2000}{%
\subsubsection{Considering Industrial Production of India after
2000}\label{considering-industrial-production-of-india-after-2000}}

\begin{Shaded}
\begin{Highlighting}[]
\NormalTok{industrial\_prod }\OtherTok{\textless{}{-}} \FunctionTok{window}\NormalTok{(iip, }\AttributeTok{start =} \DecValTok{2000}\NormalTok{)}
\end{Highlighting}
\end{Shaded}

\hypertarget{convert-gregorian-dates-of-diwali-holidays-to-vedic-calendar-dates}{%
\subsection{Convert Gregorian dates of Diwali holidays to Vedic calendar
dates}\label{convert-gregorian-dates-of-diwali-holidays-to-vedic-calendar-dates}}

\begin{Shaded}
\begin{Highlighting}[]
\NormalTok{get\_vedic\_date}\OtherTok{\textless{}{-}} \ControlFlowTok{function}\NormalTok{(julian\_day, place) \{}
  
\NormalTok{masa\_num }\OtherTok{\textless{}{-}}\NormalTok{ VedicDateTime}\SpecialCharTok{::}\FunctionTok{masa}\NormalTok{(julian\_day, place)}
\NormalTok{vikram\_samvatsara }\OtherTok{=}\NormalTok{ VedicDateTime}\SpecialCharTok{::}\FunctionTok{elapsed\_year}\NormalTok{(julian\_day, masa\_num)[}\DecValTok{5}\NormalTok{]}
\NormalTok{tithi\_ }\OtherTok{=} \FunctionTok{tithi}\NormalTok{(julian\_day, place)[}\DecValTok{1}\NormalTok{]}
\NormalTok{masa\_ }\OtherTok{=} \FunctionTok{masa}\NormalTok{(julian\_day, place)[}\DecValTok{1}\NormalTok{]}
\NormalTok{vedic\_dates }\OtherTok{=} \FunctionTok{paste}\NormalTok{(}\FunctionTok{as.character}\NormalTok{(vikram\_samvatsara),}\StringTok{"{-}"}\NormalTok{,}\FunctionTok{as.character}\NormalTok{(masa\_),}\StringTok{"{-}"}\NormalTok{,}\FunctionTok{as.character}\NormalTok{(tithi\_), }\AttributeTok{sep =} \StringTok{""}\NormalTok{) }
\FunctionTok{return}\NormalTok{(vedic\_dates)}
\NormalTok{\}}
\end{Highlighting}
\end{Shaded}

\hypertarget{converted-diwali-dates-to-vedic-datetime-compatible-date-times}{%
\subsubsection{Converted Diwali dates to vedic datetime compatible
date-times}\label{converted-diwali-dates-to-vedic-datetime-compatible-date-times}}

\begin{Shaded}
\begin{Highlighting}[]
\NormalTok{date}\OtherTok{\textless{}{-}}\NormalTok{ seasonal}\SpecialCharTok{::}\NormalTok{diwali}
\NormalTok{date }\OtherTok{\textless{}{-}} \FunctionTok{as.POSIXct.Date}\NormalTok{(date)}
\NormalTok{julianday }\OtherTok{\textless{}{-}}\NormalTok{ insol}\SpecialCharTok{::}\FunctionTok{JD}\NormalTok{(date)}


\NormalTok{place }\OtherTok{\textless{}{-}} \FunctionTok{c}\NormalTok{(}\FloatTok{15.34}\NormalTok{, }\FloatTok{75.13}\NormalTok{, }\SpecialCharTok{+}\FloatTok{5.5}\NormalTok{) }\CommentTok{\#Latitude, Longitude and timezone of the location}
\NormalTok{diwali\_vedic\_calendar }\OtherTok{=} \FunctionTok{c}\NormalTok{()}

\ControlFlowTok{for}\NormalTok{ (i }\ControlFlowTok{in} \DecValTok{1}\SpecialCharTok{:}\FunctionTok{length}\NormalTok{(julianday)) }
\NormalTok{\{}
\NormalTok{  diwali\_vedic\_calendar }\OtherTok{=} \FunctionTok{c}\NormalTok{(diwali\_vedic\_calendar, }\FunctionTok{get\_vedic\_date}\NormalTok{(julianday[i], place))}
\NormalTok{\}}
\end{Highlighting}
\end{Shaded}

\hypertarget{generate-time-series-based-on-genhol-function-using-dates-of-diwali-as-input}{%
\subsection{Generate time-series based on `genhol()` function using
dates of Diwali as
input}\label{generate-time-series-based-on-genhol-function-using-dates-of-diwali-as-input}}

\hypertarget{generate-time-series-based-on-vedic-calendar-dates-of-diwali}{%
\subsubsection{Generate time-series based on Vedic calendar dates of
Diwali}\label{generate-time-series-based-on-vedic-calendar-dates-of-diwali}}

\begin{Shaded}
\begin{Highlighting}[]
\NormalTok{pre\_diwali\_ts\_vedic }\OtherTok{\textless{}{-}} \FunctionTok{genhol}\NormalTok{(}\FunctionTok{as.Date}\NormalTok{(diwali\_vedic\_calendar), }\AttributeTok{start =} \SpecialCharTok{{-}}\DecValTok{6}\NormalTok{, }\AttributeTok{end =} \SpecialCharTok{{-}}\DecValTok{1}\NormalTok{, }\AttributeTok{center =} \StringTok{"mean"}\NormalTok{)}
\NormalTok{post\_diwali\_ts\_vedic }\OtherTok{\textless{}{-}} \FunctionTok{genhol}\NormalTok{(}\FunctionTok{as.Date}\NormalTok{(diwali\_vedic\_calendar), }\AttributeTok{start =} \DecValTok{0}\NormalTok{, }\AttributeTok{end =} \DecValTok{6}\NormalTok{, }\AttributeTok{center =} \StringTok{"mean"}\NormalTok{)}
\end{Highlighting}
\end{Shaded}

\texttt{pre\_diwali\_ts} and \texttt{post\_diwali\_ts} both are of
time-series class object and represent 2 time-series to include pre and
post festival for better seasonal adjustment.

\hypertarget{generate-time-series-with-gregorian-calendar-system-for-diwali}{%
\paragraph{Generate time-series with Gregorian calendar system for
Diwali}\label{generate-time-series-with-gregorian-calendar-system-for-diwali}}

\begin{Shaded}
\begin{Highlighting}[]
\NormalTok{pre\_diwali\_ts}\OtherTok{\textless{}{-}} \FunctionTok{genhol}\NormalTok{(seasonal}\SpecialCharTok{::}\NormalTok{diwali, }\AttributeTok{start =} \SpecialCharTok{{-}}\DecValTok{6}\NormalTok{, }\AttributeTok{end =} \SpecialCharTok{{-}}\DecValTok{1}\NormalTok{, }\AttributeTok{center =} \StringTok{"mean"}\NormalTok{)}
\NormalTok{post\_diwali\_ts}\OtherTok{\textless{}{-}} \FunctionTok{genhol}\NormalTok{(seasonal}\SpecialCharTok{::}\NormalTok{diwali, }\AttributeTok{start =} \DecValTok{0}\NormalTok{, }\AttributeTok{end =} \DecValTok{6}\NormalTok{, }\AttributeTok{center =} \StringTok{"mean"}\NormalTok{)}
\FunctionTok{ts.plot}\NormalTok{(pre\_diwali\_ts,}\AttributeTok{lwd =} \FunctionTok{c}\NormalTok{(}\DecValTok{2}\NormalTok{, }\DecValTok{1}\NormalTok{),  }\AttributeTok{gpars=}\FunctionTok{list}\NormalTok{(}\AttributeTok{xlab=}\StringTok{"Time"}\NormalTok{, }\AttributeTok{ylab=}\StringTok{"Pre Diwali Time{-}series"}\NormalTok{, }\AttributeTok{main =} \StringTok{"Pre Diwali Indian Industrial Input Indiactor"}\NormalTok{))}
\end{Highlighting}
\end{Shaded}

\includegraphics{regressors_of_diwali_seasonality_for_industrial_production_files/figure-latex/unnamed-chunk-5-1.pdf}

\begin{Shaded}
\begin{Highlighting}[]
\FunctionTok{ts.plot}\NormalTok{(post\_diwali\_ts,}\AttributeTok{lwd =} \FunctionTok{c}\NormalTok{(}\DecValTok{2}\NormalTok{, }\DecValTok{1}\NormalTok{), }\AttributeTok{gpars=}\FunctionTok{list}\NormalTok{(}\AttributeTok{xlab=}\StringTok{"Time"}\NormalTok{, }\AttributeTok{ylab=}\StringTok{"Post Diwali Time{-}series"}\NormalTok{, }\AttributeTok{main =} \StringTok{"Post Diwali Indian Industrial Input Indiactor"}\NormalTok{))}
\end{Highlighting}
\end{Shaded}

\includegraphics{regressors_of_diwali_seasonality_for_industrial_production_files/figure-latex/unnamed-chunk-6-1.pdf}
\#\#\# Including user defined regressors

\hypertarget{converting-india-industrial-output-time-series-object-to-vedic-calendar-system}{%
\subsubsection{Converting India Industrial Output time-series object to
Vedic Calendar
system}\label{converting-india-industrial-output-time-series-object-to-vedic-calendar-system}}

Converting `seasonal::iip` ts object to Vedic calendar based time-series
object to model seasonality\\

\begin{Shaded}
\begin{Highlighting}[]
\NormalTok{iip\_data }\OtherTok{\textless{}{-}} \FunctionTok{data.frame}\NormalTok{(}\AttributeTok{Y=}\FunctionTok{as.matrix}\NormalTok{(seasonal}\SpecialCharTok{::}\NormalTok{iip), }\AttributeTok{date=}\FunctionTok{as.Date}\NormalTok{(zoo}\SpecialCharTok{::}\FunctionTok{as.yearmon}\NormalTok{(}\FunctionTok{time}\NormalTok{(seasonal}\SpecialCharTok{::}\NormalTok{iip))))}
\NormalTok{date}\OtherTok{\textless{}{-}}\NormalTok{ iip\_data}\SpecialCharTok{$}\NormalTok{date}
\NormalTok{date }\OtherTok{\textless{}{-}} \FunctionTok{as.POSIXct.Date}\NormalTok{(date)}
\NormalTok{julianday\_iip }\OtherTok{\textless{}{-}}\NormalTok{ insol}\SpecialCharTok{::}\FunctionTok{JD}\NormalTok{(date)}
\NormalTok{place }\OtherTok{\textless{}{-}} \FunctionTok{c}\NormalTok{(}\FloatTok{15.34}\NormalTok{, }\FloatTok{75.13}\NormalTok{, }\SpecialCharTok{+}\FloatTok{5.5}\NormalTok{) }\CommentTok{\#Latitude, Longitude and timezone of the location}
\NormalTok{iip\_vedic\_calendar }\OtherTok{=} \FunctionTok{c}\NormalTok{()}
\ControlFlowTok{for}\NormalTok{ (i }\ControlFlowTok{in} \DecValTok{1}\SpecialCharTok{:}\FunctionTok{length}\NormalTok{(julianday\_iip)) }
\NormalTok{\{}
\NormalTok{iip\_vedic\_calendar }\OtherTok{=} \FunctionTok{c}\NormalTok{(iip\_vedic\_calendar, }\FunctionTok{get\_vedic\_date}\NormalTok{(julianday\_iip[i], place))}
\NormalTok{\}}
\NormalTok{iip\_data}\SpecialCharTok{$}\NormalTok{date }\OtherTok{\textless{}{-}}\NormalTok{ iip\_vedic\_calendar}
\NormalTok{iip\_vedic\_data\_ts }\OtherTok{\textless{}{-}} \FunctionTok{ts}\NormalTok{(iip\_data}\SpecialCharTok{$}\NormalTok{Y,}\AttributeTok{start=}\FunctionTok{c}\NormalTok{(}\DecValTok{2061}\NormalTok{,}\DecValTok{2}\NormalTok{))}
\end{Highlighting}
\end{Shaded}

The \texttt{seasonal} package allows to add user-defined regressors to
remove seasonality from a time-series. Here \texttt{pre\_diwali\_ts} and
\texttt{post\_diwali\_ts} are added in the main seasonal adjustment.
\texttt{X-13ARIMA-SEATS} is used to adjust for the festival seasonal
component.

\begin{Shaded}
\begin{Highlighting}[]
\NormalTok{m1 }\OtherTok{\textless{}{-}} \FunctionTok{seas}\NormalTok{(industrial\_prod, }\AttributeTok{xreg =} \FunctionTok{cbind}\NormalTok{(pre\_diwali\_ts, post\_diwali\_ts), }\AttributeTok{regression.usertype =} \StringTok{"holiday"}\NormalTok{, }\AttributeTok{x11 =} \FunctionTok{list}\NormalTok{())}
\end{Highlighting}
\end{Shaded}

\texttt{xreg} adds the user-defined regressors and \texttt{x11} is
chosen as the decomposition effect.

\begin{Shaded}
\begin{Highlighting}[]
\FunctionTok{summary}\NormalTok{(m1)}
\end{Highlighting}
\end{Shaded}

\begin{verbatim}
## 
## Call:
## seas(x = industrial_prod, xreg = cbind(pre_diwali_ts, post_diwali_ts), 
##     regression.usertype = "holiday", x11 = list())
## 
## Coefficients:
##                     Estimate Std. Error z value Pr(>|z|)    
## xreg1             -0.0034498  0.0091813  -0.376  0.70711    
## xreg2             -0.0383834  0.0081358  -4.718 2.38e-06 ***
## Weekday            0.0012087  0.0004275   2.827  0.00469 ** 
## Constant          -0.0014297  0.0003227  -4.431 9.39e-06 ***
## LS2008.Nov        -0.0738919  0.0160213  -4.612 3.99e-06 ***
## MA-Nonseasonal-01  0.4328295  0.0785625   5.509 3.60e-08 ***
## MA-Seasonal-12     0.9995753  0.0785461  12.726  < 2e-16 ***
## ---
## Signif. codes:  0 '***' 0.001 '**' 0.01 '*' 0.05 '.' 0.1 ' ' 1
## 
## X11 adj.  ARIMA: (0 1 1)(0 1 1)  Obs.: 116  Transform: log
## AICc: 543.2, BIC: 562.7  QS (no seasonality in final):    0  
## Box-Ljung (no autocorr.): 31.53   Shapiro (normality): 0.9894
\end{verbatim}

The seasonal co-efficient shows minor decline during pre and post Diwali
season across the time-series. In the below non-adjusted v adjusted
seasonal plot it can be observed that seasonal adjustment based on
Diwali season removes distortion from the time-series.

\begin{Shaded}
\begin{Highlighting}[]
\FunctionTok{plot}\NormalTok{(m1, }\AttributeTok{xlab=}\StringTok{"Year"}\NormalTok{, }\AttributeTok{ylab=}\StringTok{"Indicator"}\NormalTok{, }\AttributeTok{main =} \StringTok{" Non{-}adjusted v Adjusted Seasonal Distribution"}\NormalTok{)}
\FunctionTok{lines}\NormalTok{(iip\_vedic\_data\_ts, }\AttributeTok{col=}\StringTok{"red"}\NormalTok{)}
\FunctionTok{legend}\NormalTok{(}\AttributeTok{x =} \StringTok{"bottomright"}\NormalTok{,          }
       \AttributeTok{legend =} \FunctionTok{c}\NormalTok{(}\StringTok{"Non{-}Adjusted"}\NormalTok{, }\StringTok{"Adjusted"}\NormalTok{),}
       \AttributeTok{lty =} \FunctionTok{c}\NormalTok{(}\DecValTok{1}\NormalTok{, }\DecValTok{2}\NormalTok{),           }
       \AttributeTok{col =} \FunctionTok{c}\NormalTok{(}\StringTok{"black"}\NormalTok{, }\StringTok{"red"}\NormalTok{),}
       \AttributeTok{lwd =} \DecValTok{2}\NormalTok{)       }
\end{Highlighting}
\end{Shaded}

\includegraphics{regressors_of_diwali_seasonality_for_industrial_production_files/figure-latex/unnamed-chunk-10-1.pdf}

\hypertarget{comparing-the-series}{%
\subsubsection{Comparing the series}\label{comparing-the-series}}

\begin{Shaded}
\begin{Highlighting}[]
\NormalTok{m2 }\OtherTok{\textless{}{-}} \FunctionTok{seas}\NormalTok{(}\AttributeTok{x =}\NormalTok{ iip, }\AttributeTok{x11 =} \FunctionTok{list}\NormalTok{(), }\AttributeTok{regression.variables =} \FunctionTok{c}\NormalTok{(}\StringTok{"td1coef"}\NormalTok{, }\StringTok{"ls2008.Nov"}\NormalTok{), }
           \AttributeTok{arima.model =} \StringTok{"(0 1 1)(0 1 1)"}\NormalTok{, }\AttributeTok{regression.aictest =} \ConstantTok{NULL}\NormalTok{, }\AttributeTok{outlier =} \ConstantTok{NULL}\NormalTok{, }
           \AttributeTok{transform.function =} \StringTok{"log"}\NormalTok{)}
\FunctionTok{ts.plot}\NormalTok{(}\FunctionTok{diff}\NormalTok{(}\FunctionTok{log}\NormalTok{(}\FunctionTok{cbind}\NormalTok{(}\FunctionTok{final}\NormalTok{(m1), }\FunctionTok{final}\NormalTok{(m2)))), }\AttributeTok{col =} \FunctionTok{c}\NormalTok{(}\StringTok{"red"}\NormalTok{, }\StringTok{"blue"}\NormalTok{), }\AttributeTok{lwd =} \FunctionTok{c}\NormalTok{(}\DecValTok{2}\NormalTok{, }\DecValTok{1}\NormalTok{),  }\AttributeTok{gpars=}\FunctionTok{list}\NormalTok{(}\AttributeTok{xlab=}\StringTok{"Year"}\NormalTok{, }\AttributeTok{ylab=}\StringTok{"Indicator"}\NormalTok{, }\AttributeTok{main=}\StringTok{"Comparing between 2 seasonal log adjusted series"}\NormalTok{))}
\FunctionTok{legend}\NormalTok{(}\AttributeTok{x =} \StringTok{"bottomright"}\NormalTok{,          }
       \AttributeTok{legend =} \FunctionTok{c}\NormalTok{(}\StringTok{"Non{-}Adjusted"}\NormalTok{, }\StringTok{"Adjusted"}\NormalTok{),}
       \AttributeTok{lty =} \FunctionTok{c}\NormalTok{(}\DecValTok{1}\NormalTok{, }\DecValTok{2}\NormalTok{),           }
       \AttributeTok{col =} \FunctionTok{c}\NormalTok{(}\StringTok{"blue"}\NormalTok{, }\StringTok{"red"}\NormalTok{),}
       \AttributeTok{lwd =} \DecValTok{2}\NormalTok{) }
\end{Highlighting}
\end{Shaded}

\includegraphics{regressors_of_diwali_seasonality_for_industrial_production_files/figure-latex/unnamed-chunk-11-1.pdf}

In the above chart, non-adjusted(blue) vs adjusted(red) seasonal plot
clearly shows the amount of distortion present in the series.

The below chart also indicated a level of distortion present for
industrial output due to Diwali festival.

\begin{Shaded}
\begin{Highlighting}[]
\FunctionTok{ts.plot}\NormalTok{(}\FunctionTok{final}\NormalTok{(m1), }\FunctionTok{final}\NormalTok{(m2), }\AttributeTok{col =} \FunctionTok{c}\NormalTok{(}\StringTok{"red"}\NormalTok{, }\StringTok{"blue"}\NormalTok{), }\AttributeTok{lwd =} \FunctionTok{c}\NormalTok{(}\DecValTok{2}\NormalTok{, }\DecValTok{1}\NormalTok{), }\AttributeTok{gpars=}\FunctionTok{list}\NormalTok{(}\AttributeTok{xlab=}\StringTok{"Year"}\NormalTok{, }\AttributeTok{ylab=}\StringTok{"Indicator"}\NormalTok{, }\AttributeTok{main=}\StringTok{"Comparing between 2 seasonal series"}\NormalTok{))}
\FunctionTok{legend}\NormalTok{(}\AttributeTok{x =} \StringTok{"bottomright"}\NormalTok{,          }
       \AttributeTok{legend =} \FunctionTok{c}\NormalTok{(}\StringTok{"Non{-}Adjusted"}\NormalTok{, }\StringTok{"Adjusted"}\NormalTok{),}
       \AttributeTok{lty =} \FunctionTok{c}\NormalTok{(}\DecValTok{1}\NormalTok{, }\DecValTok{2}\NormalTok{),           }
       \AttributeTok{col =} \FunctionTok{c}\NormalTok{(}\StringTok{"blue"}\NormalTok{, }\StringTok{"red"}\NormalTok{),}
       \AttributeTok{lwd =} \DecValTok{2}\NormalTok{) }
\end{Highlighting}
\end{Shaded}

\includegraphics{regressors_of_diwali_seasonality_for_industrial_production_files/figure-latex/unnamed-chunk-12-1.pdf}

\hypertarget{multi-seasonality-decomposition-for-india-industrial-output}{%
\subsection{Multi-seasonality decomposition for India industrial
output}\label{multi-seasonality-decomposition-for-india-industrial-output}}

(\textbf{forecast?}) library was used to identify multiple seasonality
present in the data-set and understand the difference between them.

\hypertarget{seasonality-decomposition-for-industrial-output-for-vedic-calendar}{%
\subsubsection{Seasonality decomposition for Industrial Output for Vedic
Calendar}\label{seasonality-decomposition-for-industrial-output-for-vedic-calendar}}

The start of the series is 1921, according to Vedic calendar system
which is equivalent to 2000

\begin{Shaded}
\begin{Highlighting}[]
\NormalTok{iip\_tbats\_vedic }\OtherTok{\textless{}{-}}\NormalTok{ forecast}\SpecialCharTok{::}\FunctionTok{msts}\NormalTok{(iip\_vedic\_data\_ts, }\AttributeTok{start=}\DecValTok{1921}\NormalTok{, }\AttributeTok{seasonal.periods =} \FunctionTok{c}\NormalTok{(}\DecValTok{2}\NormalTok{))}
\NormalTok{fit\_vedic }\OtherTok{\textless{}{-}}\NormalTok{ forecast}\SpecialCharTok{::}\FunctionTok{tbats}\NormalTok{(iip\_tbats\_vedic)}
\FunctionTok{plot}\NormalTok{(fit\_vedic)}
\end{Highlighting}
\end{Shaded}

\includegraphics{regressors_of_diwali_seasonality_for_industrial_production_files/figure-latex/unnamed-chunk-13-1.pdf}

\hypertarget{refs}{}
\begin{CSLReferences}{1}{0}
\leavevmode\vadjust pre{\hypertarget{ref-seasonal}{}}%
Sax, Christoph, and Dirk Eddelbuettel. 2018. {``Seasonal Adjustment by
{\textbraceleft}x-13ARIMA-SEATS{\textbraceright} in
{\textbraceleft}r{\textbraceright}''} 87.
\url{https://doi.org/10.18637/jss.v087.i11}.

\end{CSLReferences}

\end{document}
